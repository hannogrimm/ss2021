\documentclass[a4paper]{article}
\title{%
Self-Reflection on SE\_25 Data Science \\
  \large Spring Semester 2021}
\author{Hanno Grimm\\hanno.grimm@code.berlin}
\date{May 3, 2021}


\begin{document}
\maketitle

\newpage 
\tableofcontents
\newpage 
\section{Context}
Over the course of last semester, I developed an interest into the concepts and mechanisms of the stock market. 
Working with financial data and deriving a deeper knowledge about this domain of Applied Finances seemed like a great idea to learn more about it.
Being curious to try it, I identified that my efforts will likely go into the direction of Data Science.
Therefore, I have set Data Science to be the main focus of my studies in order to learn the exploration of (financial) data.
Data Science was a discipline I haven't had invested much of my time into researching or learning about before. I was certain that a combination of Data Science and Applied Finance in my studies will deliver a semester fruitful of curiosity-driven learnings.
\\\newline
In the following, I will reflect on my progress over the semester from a resolving end-of-semester perspective. 
\section{Goals}
The requirement for completed projects for the module assessment was at first something I had to accustome myself to, as for most of my assessments in the past where directly covered by my project work. 
I therefore had to take my time and reflect to ask myself the questions: What do I expect to learn from them? How can I find the best fit of projects to aid me in my project scope while simultaneously achieve the necessary foundations of Data Science? 
Does my project feature the right scope to learn "proper" Data Science?
\\\newline
I concluded two things for myself: I want to learn and practice foundations of financial analysis, i.e. risk management, portfolio analysis, time-series plotting. 
Meanwhile, I set am emphasis on the necessity to undergo one common Data Science Beginner path so I do not miss out foundations of Data Science in my "finance bubble", as this domain can vary quite far from other Data-related practices.
\\\newline
It was important for me to practice what I learn in projects, so I can see if I can appropriately recall the theory. 
Practice has been the greatest catalyst to master a topic for me in the past and I therefore value it in my learning paths a lot.
\\\newline
For this solo semester-project, I set out to be confident to say "I know how to approach an analysis on this dataset", i.e build a relevent knowledge base to be familiar and comfortable to take on Data Analysis in the future. 
Besides that, educating myself about stock finances was something I felt a genuine interest in. The projects main objective is to serve as a gate to exploring and understanding more of the dynamics of the stock market, its potential flaws. 
To be able to form a more informed opinion on financial topics as the real-estate crash foregoing the financial crisis in 2008\/9 - or to simply \textit{understand them} - is eminently important for me to achieve with my efforts and therefore my target goal.
\section{Execution \& Achievements}
Regarding a general approach to learning: 
Sitting down and preparing a semester plan was really helpful to keep track of the necessary steps to achieve my goals. 
I was able to see when I was lacking behind with my estimated timeline, which was also a great chance to reflect on what my priorities should be for the recent time.
\\\newline
For my own project, well, I have clearly underestimated the difficulty of gathering the relevant data and the intensity on forming a proper data analysis. My initial plan to do a Data Analysis on the market manipulation patterns on the GameStop stock were completely dropped around the mid-semester. This was simply due to the difficulty of gathering the data (behind a paywall). Furthermore, I identified that I do not have the necessary experience with data analysis to come to fruitful results until the end of the semester. Switching to a simpler project that implements the main learnings of my courses and guided projects was a great way to challenge if I have learned and comprehended the skills I found important over the course of the semester.
\\\newline
I am really happy with the outcome of my project. As I have written a fair reflection on the project itself in its notebook, I do not want to repeat myself on my specific learnings with it here. I rather want to focus on the learnings I have gathered about approaching \textit{learning} itself: 

A plan in mind is key.  I have gone through three iterations of my project and I am confident to say that the first iteration failed to preserve due to lacking plan and scope for my analysis attempts. 
The guide on Notion on how to approach one's own data analysis was really helpful. I am not used to sitting down and spending \textit{a day} to plan my next weeks. Today, I am really happy I did so and I find the value in mapping out the road ahead, \textit{especially} because I am not used to do so.

My second learning is: Take it slow, my friend. I had great ambitions for jumping into this new domain. From my advanced experience with other programming tasks, I gained the necessary knowledge and experience to build "great things". Diving into Data Science has been an educational process that made me realize again that gaining knowledge in a domain is not about learning complexity fast, but gaining fundamental insights steadily. Showing off with complexity and \textit{cool topics} can burn-out once you run into problems, because the fundamental ideas were not understood. Meanwhile, educating yourself steadily with foundations gives you the right tools to tackle any roadblock ahead!
\section{Self Evaluation}
I have not achieved my initial goals, but I am happy for it. I feel quite confident with manipulating data with pandas and visualizing it with matplotlib or seaborn. I understand the necessities of a precise and clear communication of findings and feel comfortable exploring the next upcoming dataset. I lack the experience with more advanced (and professional) data analyses. I therefore settle for a Level 1 in my self-assessment, happy to take charge for an up-level in future semesters with more experience!
\end{document}